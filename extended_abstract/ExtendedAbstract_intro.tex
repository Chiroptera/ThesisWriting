%!TEX root = ExtendedAbstract.tex
%%%%%%%%%%%%%%%%%%%%%%%%%%%%%%%%%%%%%%%%%%%%%%%%%%%%%%%%%%%%%%%%%%%%%%
%     File: ExtendedAbstract_intro.tex                               %
%     Tex Master: ExtendedAbstract.tex                               %
%                                                                    %
%     Author: Andre Calado Marta                                     %
%     Last modified : 27 Dez 2011                                    %
%%%%%%%%%%%%%%%%%%%%%%%%%%%%%%%%%%%%%%%%%%%%%%%%%%%%%%%%%%%%%%%%%%%%%%
% State the objectives of the work and provide an adequate background,
% avoiding a detailed literature survey or a summary of the results.
%%%%%%%%%%%%%%%%%%%%%%%%%%%%%%%%%%%%%%%%%%%%%%%%%%%%%%%%%%%%%%%%%%%%%%

\section{Introduction}
\label{sec:intro}

% Motivation and state-of-the-art...

% Include relevant references~.\cite{Fred2005}
% \IEEEPARstart{A}{dvances}

% Advances
\IEEEPARstart{A}{dvances} in technology allow for the collection and storage of unprecedented amount and variety of data, of which there is an interest in performing automated analysis for generation of knowledge and new insights.
% Advances in technology allow for the collection and storage of unprecedented amount and variety of data, a concept commonly designated by \emph{Big Data}.
% Most of this data is stored electronically and there is an interest in automated analysis for generation of knowledge and new insights.
A growing body of formal methods aiming to model, structure and/or classify data already exist, e.g. linear regression, principal component analysis, cluster analysis, support vector machines, neural networks.
% tecnicas de clustering como as tecnicas formais de abordar estes problemas
Cluster analysis is an interesting tool because it typically does not make assumptions on the structure of the data, which is specially interesting when no prior information about the data is known.

A vast body of work on clustering algorithms exists, but usually different methods are suited to datasets of different characteristics.
Currently, there are state of the art algorithms that are more robust than "traditional" algorithms by having a wider applicability or being less dependent on input parameters.
One such approach is Evidence Accumulation Clustering (EAC) \cite{Fred2005}, belonging to the wider class of ensemble methods.
EAC is a state-of-the art clustering method that addresses the robustness challenge, but, currently, its computational complexity restricts its application to small datasets.

The present work is concerned with pushing the current limits of the EAC method to large datasets by addressing the challenges of scalability and efficiency without compromising robustness, using technology available in a desktop workstation.
Two main approaches exist for scaling: using algorithms with better computational complexity in the EAC steps or turning to parallel and external memory computation for speeding up and addressing the space complexity.
Quantum clustering algorithms \cite{Casper2012KMeans,Horn2001a} were researched under the motivation of having better computational complexity, but it proved to be a fruitless endeavor mainly due to its prohibitive computational complexity within the EAC context.
% Research on using quantum clustering algorithms \cite{Casper2012KMeans,Horn2001a} for EAC proved fruitless mainly due to its prohibitive computational complexity within the EAC context.
This moved the focus of research to parallel computation, more specifically General Purpose computation in Graphics Processing Units (GPGPU), and external memory (using hard drives) solutions.

Selected work from the this dissertation was compiled into a conference paper and submitted to the $5^{th}$ The International Conference on Pattern Recognition Applications and Methods.

This document is structured as follows.
Relevant concepts to understand the work done are presented in section \ref{sec:backg}.
Since EAC is a three step method and each of these must be optimized individually, sections \ref{sec:production}, \ref{sec:combination} and \ref{sec:recovery} explain what was done in each step.
Finally, the implemented optimizations are tested and the results presented in section \ref{sec:resul}.

% Two approaches addressing the space complexity of EAC currently exist.
% One uses a 
